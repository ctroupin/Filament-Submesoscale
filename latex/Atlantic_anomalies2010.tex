\documentclass[12pt]{article}
\usepackage[dvips]{graphicx}
\usepackage{color}
\usepackage{amsmath}
\usepackage{amssymb}
\usepackage{float}
\usepackage{caption}
\usepackage{hyperref}
\usepackage{natbib}
\usepackage{multicol}
\usepackage{subfigure}
\RequirePackage{lineno}
\usepackage{url}
\usepackage{doi}
\usepackage{datetime}
\usepackage{fmtcount}

\hypersetup{bookmarksopen=true,
bookmarksnumbered=true,  
pdffitwindow=false, 
pdfstartview=FitH,
pdftoolbar=true,
pdfmenubar=true,
pdfwindowui=true,
pdfauthor=Charles Troupin,
bookmarksopenlevel=2,
colorlinks=true,%
breaklinks=true,%
colorlinks=true,%
linkcolor=blue,anchorcolor=blue,%
citecolor=blue,filecolor=blue,%
menucolor=blue,%
urlcolor=blue}


\bibliographystyle{agu}

\DeclareGraphicsExtensions{.pdf}
\graphicspath{
{../figures/anomalies/}
}

\title{Evidences and consequences of wind-induced temperatures anomalies in the tropical and subtropical North Atlantic Ocean in winter-spring 2010}
\author{Charles Troupin$^{1}$ \& Francisco Machín$^2$}
\date{}

\begin{document}
\linenumbers
\maketitle
\begin{center}
Last modified: \textbf{\today , \currenttime}
\end{center}

%\begin{affiliations}
% \item GeoHydrodynamics and Environment Research, AGO-MARE, University of Li\`{e}ge, Li\`{e}ge, Belgium
% \item Facultad de Ciencias del Mar, University of Las Palmas de Gran Canaria, Las Palmas, Spain
%\end{affiliations}

\begin{abstract}

During the first months of 2010, the tropical and subtropical North Atlantic displayed anomalously high temperatures, with values seldom observed during the last decades. In situ and remote sensing data are used to evaluate horizontal, vertical and temporal extensions of the anomalies. The repercussions on the seasonal evolution of the mixed layer are examined; in particular, it is shown that the northwest Africa coastal upwelling is significantly weakened in comparison to previous years. The consequences on the biological variables are examined by means of satellite-derived measurements. A simple mechanism related to changes in wind intensity is proposed in order to explain our observations. The wind weakening coincides with a strongly negative value of the North Atlantic Oscillation index. 

\end{abstract}

In the subtropical northeast Atlantic Ocean, the seasonal cycle of physical properties is mainly driven by air-sea interactions, namely wind stress and net heat flux. In winter the net heat flux reaches negative values (i.e. transfert from ocean to atmosphere), responsible for the erosion of the thermocline. A late winter phytoplankton bloom is associated to this physical process \citep[e.g.][]{DELEON73,HERNANDEZLEON84,HERNANDEZLEON04}. In summer, Trade winds intensify due to the northward motion of the Azores High \citep{WOOSTER76}, while the net heat flux reaches its maximum. The result of these two counteracting processes is a strong stratification of the ocean surface layers. A shallow mixed layer prevents the injection of nutrients from deeper waters. Recent observations depict significant changes of the described seasonal cycle during the first months of 2010. It is of primary importance to examine these changes, because of the coupling between biological and physical cycles \citep{ARISTEGUI01,TROUPIN10} and the influence of SST on the hurricane activity in the Atlantic Ocean \citep[e.g.][]{GOLDENBERG2001}. 

Remote-sensing images provide high-resolution information on the ocean surface, with an almost daily global coverage. They constitute a first tool to examine the changes of the ocean physical and biological characteristics. Sea Surface Temperature (SST) anomalies with respect to the 2002-2008 period are derived using measurements from the MODIS sensor on the Aqua satellite. SST anomalies (See Method section) are shown for winter 2010 in the north subtropical and tropical Atlantic Ocean(Fig.~\ref{fig:sst_aqua_season_anomaly_wind_4}). A dipole structure is visible, with negative anomalies in the north and positive in the south. The transition between the two areas is sharp and takes place along a line joining the Strait of Gibraltar (36$^{\circ}$~N) to the Caribbean islands (20$^{\circ}$~N). Positive anomalies are particularly high along the NW Africa coast, between 12 and 32$^{\circ}$~N. This region coincides with the Canary Current upwelling system. 

To analyze if the dipole structure of anomalies has occurred in the past three decades, NOAA Optimum Interpolation SST analysis \citep{REYNOLDS02} is considered. It is constructed using both in situ and satellite data and provides SST gridded fields back to 1981. The domain-averaged values were computed for each month until August 2010 (Fig.~\ref{fig:sst_med_reynolds_areacontrol}). The climatological cycle in the studied region is as follows: the minimal temperatures are reached in February-March, when the mixed layer is the deepest, under the effect of convective mixing. The temperature starts to increase in spring, when the net heat flux becomes positive again. Maximal values are reached in August-September and correspond to the shallowest mixed layer \citep[e.g.][]{TROUPIN10}. 

During the first half of 2009, surface temperatures were slightly below the long-term mean, with a difference not larger than 0.25$^{\circ}$C. This changes in June 2009, when temperatures become higher than the 1981-2008 average. In early 2010, the deviation with respect to the climatology  further increases: the mean value for February 2010 (23.3$^{\circ}$C) was only attained between April and May the year before. Furthermore, from March to August 2010, the monthly mean temperatures were higher than the values of any of the corresponding month during the 30 previous years. Hence the observed temperature anomalies seem to be an unusual event. 

After inspection of the anomaly recurrence in time, their vertical extension is now analyzed. In situ profiles from the World Ocean Database 2009 \citep[WOD09,][]{BOYER09} and profilers from the Global Ocean Data Assimilation Experiment (GODAE) data catalog are considered. A set of three profilers located south of the Canary archipelago was selected. They were chosen because of their location and their long lifetime. Their displacements are relatively short and principally zonal over the considered period. The seasonal cycle measured by profiler no.~4900825 (Fig.~\ref{fig:4900825_tmed_largo}) is similar to the description made previously: lowest temperatures appear in late winter under the effect of the thermocline erosion. During this period, the mixed layer frequently reaches depths well below 100~m. The computed mixed-layer depth (MLD hereinafter) 2010 displays a distinctive behavior in 2010: the values stay close to summer time MLD, around 50~m, with the deepest values not below 75~m. Again, it is confirmed that winter 2010 surface temperatures are 2$^{\circ}$C higher than past years. The mean profile of density also leads to striking observations: in winter 2010 the density was lower than spring 2010 density by about 0.5~kg/m$^{3}$, whereas the climatological profile shows the opposite. It is of particular importance, as the biological productivity is essentially concentrated in winter. Similar seasonal cycles are found with the other profilers (Supplementary Material). 

Profilers provide a good representation of the seasonal cycle of the mixed layer. It remains now to confirm the vertical extension of the area affected by the anomalies. To this end, we computed anomalies of in situ profiles extracted from the World Ocean Database with respect to the World Ocean Atlas 2009 climatology. The winter 2010 anomalies at 75~m (Fig.~\ref{fig:temp_anomaly_WOD_WOA09_13_28}) confirm the extension of the positive-anomaly area, which agrees with the structure determined by satellite images (Fig.~\ref{fig:sst_aqua_season_anomaly_wind_4})). Several points off NW Africa display anomalies up to 4$^{\circ}$C with respect the climatology, in particular between Canary and Cape Verde archipelagos.

In the Canary Current upwelling system, the biological cycles are strongly influenced by the mixed-layer evolution \citep[e.g.][]{TROUPIN10}. It is then relevant to examine satellite-derived chlorophyll-a concentrations. Winter 2010 anomalies (Fig.~\ref{fig:chloro_aqua_season_anomaly_4})  can be broken up into three subregions: the open ocean, where the anomalies are low, a narrow band of increased concentration around 32$^{\circ}$N and a coastal area off NW Africa, with strongly negative anomalies. Comparison with the SST anomaly map (Fig.~\ref{fig:sst_aqua_season_anomaly_wind_4}) highlights a correspondence between the areas of positive (negative) temperature anomalies with negative (positive) chlorophyll-a concentration anomalies. In particular, the upwelling area displays substantial decrease of the chlorophyll-a concentration, as a result of a weakened biological activity, especially between Cape Ghir (31$^{\circ}$N) and south of Cape Roxo (12$^{\circ}$N). Similar spatial distributions (not shown here) are obtained for particular organic carbon (POC) and particular inorganic carbon (PIC) measured by the MODIS sensor.

A mechanism that accounts for the observed changes is proposed hereafter. As coastal upwelling intensity depends on wind strength and orientation, wind intensity anomalies are computed in the studied area. The resulting isopleths are superimposed on the SST map (Fig.~\ref{fig:sst_aqua_season_anomaly_wind_4}): the region of negative wind anomalies shows a remarkable similitude with the region of warmer temperatures. Another possible explanation might be a variation in the net heat flux penetrating the water column. To verify this hypothesis, we examined heat flux components over the last three decades. The results (not shown here) demonstrate that the net heat flux did not undergo significant variations in order to contribute to the temperature change described here. 

However, the latent and sensible components of the net heat flux are dependent on the wind intensity \citep[e.g.][]{CHARNOCK67,KONDO75,YU04}. Hence in the area of negative wind anomaly, these components have a reduced intensity and then the lost of heat by the ocean is decreased. This possible interaction between latent and sensible heat fluxes with SST was proposed by \cite{CAYAN92}. On the contrary, the long-wave heat flux is an increasing function of the ocean temperature. Thus in the region of positive SST anomalies, this component tends to increase the lost of heat by the ocean. All in all, the variations of the heat flux components roughly compensate one each other, and the interannual variations are relatively weak.

To sum up, the mechanism we propose is as follows: in a portion of the north Atlantic Ocean, the wind intensity decreases and provokes different consequences, according to the region:

\begin{enumerate}
\item Off northwest Africa, where it is climatologically equatorward, the wind is responsible for a weakening of the coastal upwelling, resulting in warmer than normal temperature, and lower than normal chlorophyll-a concentrations.
\item In the open ocean, it causes a shallowing of the mixed layer. As a same heat flux is supplied to a smaller quantity of water (comprised between the surface and the mixed layer depth), the temperature increases. Yet, chlorophyll-a concentration anomalies are not significant is this area, because the productivity is generally low, and the changes in the wind intensity themselves do not constitute a sufficient factor to stimulate the biological activity.    
\end{enumerate}

What is still to investigate is the origin of the wind intensity modification. Since the large-scale wind structure in the studied region is notably determined by the Azores High position, it is natural to look at the evolution of the North Atlantic Oscillation (NAO) index. NAO effects on the North Atlantic Ocean are numerous, both on physical \citep{WANNER01,MARSHALL02,HURREL09} and ecological bases \citep{OTTERSEN01}. Wind and NAO indexes are plotted in Fig.~\ref{fig:monthly_winds_ecmwf_ERA40_Interim}. Their relationship has already been investigated \citep[e.g.][]{HURREL95,MARSHALL01}, but the implication on the tropical SST is not perfectly understood.

Over the past three decades, the NOA index has been mostly positive. The most notable exceptions are 1996, 2001 and 2010. The negative NAO index phase means that Azores high and Icelandic low are weaker, resulting in a lower number of winter storms, with lower intensity, and more humidity is carried to southern Europe. A negative NAO index is not always correlated to a weakening of the winds, although it is the case for the negative episodes in 1989, 1996 and 2010. The results presented here are in agreement with \cite{SANTOS05}, who showed that warm (cold) mid-latitude SST anomalies were associated to low (high) NAO in the Canary Current system. However, their analysis is limited to a smaller region and restricted to the surface layer. The link between SST and NAO was also showed by \cite{VISBECK98} with an ocean model and by \cite{HURREL09}. The latter also report a correspondence between MLD and NAO.

\section*{Conclusion}

We analyzed in situ and remote sensing data in the tropical and subtropical North Atlantic in winter and spring 2010. An extended region of positive temperature anomalies develops in the southern part of the domain of interest and reaches its maximum extension in February. A set of profilers launched in the region showed the modifications of the water column properties near the surface: the development of a mixed layer that usually takes place in winter seems to be prevented in 2010. The analysis of in situ profiles confirms the information brought by the remote sensing measurements and permits to determine the vertical extension of the anomalies.  

Wind seems to be the main factor for these anomalies: a wind intensity decrease causes a shallower mixed layer and a weaker upwelling in the coastal area of northwest Africa. The wind fields we analyzed are in agreement with this assumption. 

%Given the variety of data processed, the proposed mechanism seems reasonable. The use of a numerical model would certainly lead to a better %understanding of the situation.  

\section*{Methods}
%%%\begin{methods}
All data used in the analyses are freely available on-line. They are described in Tab.~\ref{tab:datasources}. The processing essentially consists of averaging, anomaly calculation and spatial interpolations.
 
Level~3 MODIS SST and chlorophyll-a concentration anomalies (Figs.~\ref{fig:sst_aqua_season_anomaly_wind_4} and \ref{fig:chloro_aqua_season_anomaly_4}) are computed for each month and season of 2009 and 2010 by subtracting the climatological mean computed using data from 2002 to 2008. Wind speed anomalies are computed with respect to 1989-2008 monthly averages. Monthly mean values (Fig.~\ref{fig:sst_med_reynolds_areacontrol}) are obtained by averaging NOAA~OI~SST over the area of interest. A quality control is applied to profiler data prior to their mapping (Fig.~\ref{fig:4900825_tmed_largo}). Mixed-layer depth is computed using with the criterion of 0.2$^{\circ}$C decrease with respect to the temperature at 10~m \citep{KARA00}. Climatological values are obtained by horizontal interpolation of the WOA09 gridded fields. In situ anomalies (Fig.~\ref{fig:temp_anomaly_WOD_WOA09_13_28}) are calculated by interpolating the WOD vertical profiles onto standard isobaric levels using the weighted parabolic method \citep[WPI,][]{REINIGER68} and subtracting the climatological value (WOA09) horizontally interpolated at the corresponding profile location. Wind velocities (Fig.~\ref{fig:monthly_winds_ecmwf_ERA40_Interim}) are averaged over the area of study; a Butterworth low-pass filter is applied to the wind time-series to remove high-frequency variability.
%%%\end{methods}


\bibliography{TemperatureAnomalies2010.bib}



%%\begin{addendum}
%% \item NASA Ocean Color and US NODC are acknowledged for making the data freely available, ECMWF for providing the atmospheric fields and J. Hurrel for NAO indices. Charles' thesis was supported by the National Fund for the Scientific Research (Belgium).
%% 
%% \item[Competing Interests] The authors declare that they have no
%%competing financial interests.
%%
%% \item[Correspondence] Correspondence and requests for materials
%%should be addressed to C. Troupin~(email: ctroupin@ulg.ac.be).
%%\end{addendum}

\begin{itemize}
 \item Aknowledgements: NASA Ocean Color and US NODC are acknowledged for making the remote sensing and in situ data freely available, ECMWF for providing the atmospheric fields and J. Hurrel for NAO indices. Constructive comments by A.~Alvera-Azc\'{a}rate, F.~Lenartz, J.-M.~Beckers (GHER-ULg) and M.~Rixen (NURC) contributed to improve the manuscript. Charles' thesis was supported by a FRIA grant from the the National Fund for the Scientific Research (Belgium).
 
 \item Competing Interests: The authors declare that they have no
competing financial interests.

 \item Correspondence: Correspondence and requests for materials
should be addressed to C. Troupin~(email: ctroupin@ulg.ac.be).

\end{itemize}


\begin{figure}[H]
\centering
\includegraphics[width=0.5\textwidth]{sst_aqua_season_anomaly_wind_4.eps}
\caption{Sea surface temperature and wind speed anomalies (contour lines) for winter 2010. Plain (dashed) contours lines denote positive (negative) wind anomalies.\label{fig:sst_aqua_season_anomaly_wind_4}}
\end{figure}

\begin{figure}[H]
\centering
\includegraphics[width=.5\textwidth]{sst_med_reynolds_areacontrol.eps}
\caption{Annual cycle of SST computed in the control area (shaded area on the map). Gray dots show the individual monthly averages from 1981 to 2008, blue, red and black lines represent 2009, 2010 and 1981-2008 averaged cycles, respectively.\label{fig:sst_med_reynolds_areacontrol}}
\end{figure}

\begin{figure}[H]
\centering
\includegraphics[width=.5\textwidth]{4900825_tmed_largo.eps}
\caption{Temperature measured by profiler no.~4900825. The plain gray line indicates the mixed later depth.The dashed black curves represent the climatological cycle.\label{fig:4900825_tmed_largo}}
\end{figure} 


\begin{figure}[H]
\centering
\includegraphics[width=.5\textwidth]{temp_anomaly_WOD_WOA09_13_28.eps}
\caption{Temperature anomalies with respect to WOA09 obtained for the profiles in winter 2010 at 75~m deep.\label{fig:temp_anomaly_WOD_WOA09_13_28}}
\end{figure}

\begin{figure}[H]
\centering
\includegraphics[width=0.5\textwidth]{chloro_aqua_season_anomaly_4.eps}
\caption{Chlorophyll-a concentration anomalies for winter 2010. Blue area denotes zone of decreased concentration.\label{fig:chloro_aqua_season_anomaly_4}}
\end{figure}

\begin{figure}[H]
\centering
\includegraphics[width=.75\textwidth]{monthly_winds_ecmwf_ERA40_Interim.eps}
\caption{Evolution of seasonal NAO index and wind speed from ECMWF 40-year reanalysis and interim product, from 1980 to 2010.\label{fig:monthly_winds_ecmwf_ERA40_Interim}}
\end{figure}





\begin{table}[H]
\centering
\footnotesize
\caption{Data used for the analysis. \label{tab:datasources}}
\begin{tabular}{lll}
\hline
Data type					& Product						  				& Provider	\& References\\
\hline
Remote sensing				& MODIS-Aqua L3 								& NASA Ocean Color \citep{FELDMAN10}\\
SST climatology				& OI~SST										& NOAA \citep{REYNOLDS02,SMITH08}	\\
In situ	profiles			& WOD09											& US~NODC \citep{BOYER09}			\\
Profilers				    & ARGO 											& US~GODAE							\\
Hydrographic climatology	& WOA09 temperature								& US~NODC \citep{LOCARNINI10}		\\
							& WOA09 salinity								& US~NODC \citep{ANTONOV10}			\\
Wind velocity   			& ERA-40 Re-analysis							& ECMWF	\citep{UPPALA05}			\\
							& Interim Re-analysis							& ECMWF \citep{SIMMONS06,UPPALA08}	\\
Heat fluxes					& Daily Re-analysis								& NCEP/NCAR \citep{KALNAY96}		\\
NAO index					& Seasonal station-based index					& NCAR CDG's Climate Analysis Section \citep{HURREL95} \\
\hline
\end{tabular}
\end{table}	


\end{document}

%--------------------------------------------------------------------------------------


good way to evaluate the changes with respect to previous years.  They are computed on a monthly basis for 2009 and 2010. For a given month, the anomaly is obtained by subtracting the 2003-2008 averaged monthly field to the corresponding monthly field. We pay special attention to the spatial distribution of the anomalies, more than on their amplitude. In November-December 2009, an area of anomalously warm water extends from America to Africa coasts, the anomalies being more homogeneous in December. In January-February, the area is better delimited spatially: the separation between positive and negative anomalies is approximatively along a line joining the Strait of Gibraltar to the Island of Cuba. The strongest values are located along the NW Africa coast, from 15$^{\circ}$N to 30$^{\circ}$N. These strong anomalies persists until April 2010. A possible explanation could be a weakening of the coastal upwelling. As upwelling are mainly driven by wind, it is relevant to consider wind as a responsible factor for the observed changes. Upwelling areas are often characterized by a high-biological activity, hence we can also expect to see aftermaths in the remote sensing derived parameters. In March the positive anomaly has strongly diminished, both in space and intensity. Finally in May and June, there only remains a small region in the open ocean (30$^{\circ}$N-45$^{\circ}$N) and portion of the coastal area (16$^{\circ}$N and 24$^{\circ}$N) with positive anomalies. 

\subsection{Time evolution}

In order to inspect the temporal recurrence of such anomalies during the last decades, Reynolds SST is averaged over the studied area (Fig.~\ref{fig:sst_med_reynolds_areacontrol}). Even is this product has a lower resolution than MODIS, it is preferred for analyzing time evolution as it is available from 1981. The climatological seasonal cycle in the control area is as following: in winter, the ocean loses heat and undergoes mixing by convection. The minimal temperature is reached in February-March, when the mixed layer is the deepest. In spring, the net heat flux is again positive, the ocean starts to stratify. The maximal temperature is reached in August-September. During the first half of 2009, the surface temperature was slightly lower than the long-term mean, with a maximal difference about 0.25$^{\circ}$C. From June 2009 to June 2010, temperatures are higher than the 1981-2008 average. The differences between the two series are particularly high during late winter and spring 2010: around 0.5$^{\circ}$C warmer than the long-term mean and around 1$^{\circ}$C warmer than the corresponding months in 2009. Furthermore, from March to June, the mean temperatures are the highest of the whole time series.


\subsection{Depth structure}

The depth extension of the anomalies is studied by the mean of in situ profiles extracted from the WOD09 and profilers from the USGODAE data catalog. The profiler allow us to represent the annual cycle in 2009-2010 and compare it to the conditions corresponding to previous years (Fig.~\ref{fig:temp_anomaly_WOD_WOA09}). Since the displacements of the profiles are relatively short and principally zonal over the considered period, we assume that the changes cannot be attributed to a change of latitude. Figure~\ref{fig:temp_anomaly_WOD_WOA09} does not only highlights the difference of temperature near surface, but also the absence of a mixed layer between January and April. The density measured by the profilers is also compared to the climatological profiles extracted from the WOA09 (Fig.~\ref{fig:temp_anomaly_WOD_WOA09}a-c, right panels). It turns out that densities in 2010 are lower than climatological values, a sign of a continued stratification during the first half of the year. It is also surprising to note that for profiler n$^{\circ}$6900610, winter density is lower than in spring, whereas climatology indicates the opposite.

The study of this set of profilers provides us a good representation of the seasonal cycle of the mixed layer. It remains now to determine the area affected by the anomalies. To this end, we computed the anomalies of in situ profiles from the WOD09 with respect to the WOA09. Each profile is vertically interpolated on the isobaric levels using the weighted parabolic method described by \cite{REINIGER68}. The climatological values at the locations of the profiles are obtained by a bilinear interpolation applied to the WOA09 fields. The anomalies at 75~m in March 2010 (Fig.~\ref{fig:temp_anomaly_WOD_WOA09}d) confirms the location of the positive-anomaly area, in agreement with satellite images, and its vertical range. Several points off NW Africa display anomalies up to 4$^{\circ}$C with respect the climatology, in particular near Canary and Cape Verde archipelagos. 
 

\subsection{Biological implication}

In the tropical North Atlantic, the phytoplankton bloom is driven by the deepening of the mixed layer in February-March \citep[e.g.][]{TROUPIN10}. The absence of development of a mixed layer in winter 2010 is expected to affect the biological cycle. Monthly chlorophyll-a concentration anomalies are presented in Fig.~\ref{fig:chlorophyll_modis_aqua_month_anomaly}. The area of negative anomaly of concentrations corresponds quite well to the region of positive temperature anomaly. In the open ocean the anomalies are weaker, but usually the production is much lower there than in coastal area, hence the small differences. The strongest anomalies are visible in the coastal upwelling area, . The offshore extension of the negative anomaly region is maximal in February and March and progressively decreases in May and June. From January to April, a band of positive anomaly is present in the area that corresponds to the negative temperature anomaly. The bands becomes thinner in May and extends over a limited region between 45$^{\circ}$N and 60$^{\circ}$N in June. 



