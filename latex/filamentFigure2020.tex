\documentclass[final,table,svgnames]{article}
\usepackage{graphicx}
\usepackage{xcolor}
\usepackage{subcaption}
\usepackage{hyperref}
\usepackage[margin=0.15in]{geometry}
\usepackage{fontspec}
\setmainfont{Carlito}



\hypersetup{bookmarksopen=true,
bookmarksnumbered=true,  
pdffitwindow=false, 
pdfstartview=FitH,
pdftoolbar=false,
pdfmenubar=false,
pdfwindowui=true,
pdfauthor=Charles Troupin,
pdftitle=Illustrations of upwelling filaments,
colorlinks=true,%
breaklinks=true,%
linkcolor=blue,anchorcolor=blue,%
citecolor=blue,filecolor=blue,%
menucolor=blue,%
urlcolor=blue}


\defaultfontfeatures{Ligatures=TeX}

   
% ------------------------------------------------
% LENGTH DEFINITIONS
%-------------------------------------------------
%\setlength{\textwidth}{24cm}
%\setlength{\textheight}{28cm}

\DeclareGraphicsExtensions{.eps,.JPG,.jpg,.pdf,.png,.PNG,.jpeg}
\graphicspath{../figures/}


\begin{document}
\pagestyle{empty}

\begin{figure}
\centering
\begin{subfigure}[t]{\textwidth}
\caption{20-year averaged sea-surface temperature and wind vectors in the four Eastern Boundary Current Upwelling Systems.}
\includegraphics[width=\textwidth]{../figures/SST_wind_summer_2018_07_006.png}
\end{subfigure}

%\begin{subfigure}[t]{.55\textwidth}
%\caption{Temporal and spatial scales of oceanic processes.}
%\includegraphics[width=\textwidth]{../figures/oceanscales.png}
%\end{subfigure}

%\begin{subfigure}[t]{.5\textwidth}
%\caption{Mean SST in summer 2018 in the Canary/Iberia upwelling system.}
%\includegraphics[width=\textwidth]{../figures/SST_summer2018_Canbus.png}
%\end{subfigure}

\begin{subfigure}[t]{.75\textwidth}
\caption{SST on January 1, 2020 off Northwest Africa. The coastal upwelling exhibits irregularities and a well-developed filament at 31$^{\circ}$N. Data from VIIRS sensor onboard SNPP satellite}
\includegraphics[width=\textwidth]{../figures/SNPP_VIIRS-20200101T022400-L2-SST-NRT.png}
\end{subfigure}

\end{figure}

\begin{figure}[t]\ContinuedFloat
\centering
\begin{subfigure}[t]{.75\textwidth}
\caption{Closer view of the previous figure showing he filament and its complex structure.}
\includegraphics[width=\textwidth]{../figures/SNPP_VIIRS-20200101T022400-L2-SST-NRT_zoom.png}
\end{subfigure}

\begin{subfigure}[t]{.45\textwidth}
\caption{QuikSCAT wind on August 15, 2009, showing the wind speed decrease over colder water.}
\includegraphics[width=\textwidth]{../figures/wind/CapeGhir-qs_l2b_52896_v4.png}
\end{subfigure}\hspace*{.25cm}
\begin{subfigure}[t]{.45\textwidth}
\caption{A phytoplankton bloom in the Benguela upwelling system reveals complex patters Aqua/MODIS image, September 2, 2017. Source: OceanColor \url{https://oceancolor.gsfc.nasa.gov/}}
\includegraphics[width=\textwidth]{../figures/A2017245125500_BenguelaUpwelling.jpg}
\end{subfigure}




\end{figure}



\end{document}
